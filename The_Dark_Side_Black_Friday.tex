\pdfminorversion=4
\documentclass[]{article}
\title {O lado sombrio da Black Friday ou The Dark Side of the Black Friday}
\begin{document}

%%%%%%%%%%%%%%%%%%%%%%%%%%%%%%%%%%%%%%%%%%%%%%%%%%%%%%%%%%%%%%%%%%%%%%%%
% planejar o orçamento
% planejar a lista de compras
% pesquisar o histórico de preços (zoom, buscape)
% priorize o cartão de crédito
% verifique o limite do cartão
% reputação das empresas (reclameaqui, procon-sp)
% atenção ao valor do frete e ao prazo de entrega (valor produto + frete)
% phishing
% ssl
% uso de rede segura (em sua casa de preferência)
% documente tudo (para facilitar o ressarcimento + print da tela com o resumo da compra)
% Cancelamento 7 dias (CDC art 49)
% Instabilidade de tráfego (promoções de quinta a segunda)
% janeiro os preços caem
% http://g1.globo.com/economia/seu-dinheiro/noticia/2016/11/veja-como-aproveitar-os-descontos-da-black-friday-sem-cair-em-roubadas.html
% http://www.history.com/news/whats-the-real-history-of-black-friday
% https://en.wikipedia.org/wiki/Black_Friday_(shopping)#Origin_of_the_term
% https://www.zoom.com.br/
% http://www.buscape.com.br
% https://www.reclameaqui.com.br/
% http://sistemas.procon.sp.gov.br/evitesite/list/evitesites.php
%%%%%%%%%%%%%%%%%%%%%%%%%%%%%%%%%%%%%%%%%%%%%%%%%%%%%%%%%%%%%%%%%%%%%%
\maketitle

Na sua sexta edição brasileira, a Black Friday um evento internacional de compras, que ocorre na quarta sexta-feira do mês de novembro, um dia após a comemoração do feriado de Ação de Graças nos E.U.A. (dia de agradecer por tudo de bom que acontecera naquele ano) e dá início a temporada de compras do Natal.\\
\newline
O comércio atribui a origem do nome do evento ao dia em que os comerciantes saem do vermelho, ou seja, deixam de usar a tinta vermelha (que representa prejuízo) nos seus livros-caixas e passam a usar a tinta preta (que representa lucro).\\
\newline
Porém historicamente a origem do nome é um pouco mais sombria:\\
\newline
Segundo o site History, na década de 50 aconteceu uma verdadeira invasão das ruas da Filadélfia por hordas de compradores e turistas, antes do jogo de futebol americano da Marinha que aconteceria no sábado. Este dia, foi de muito trabalho e problemas para os policias da Filadélfia que teve de lidar com roubos e confusões nos comércios e nas ruas daquela cidade, este dia foi apelidado de Black Friday. Outras histórias também são dadas como origem para o nome: A Black Friday seria o dia em que os comerciantes de escravos davam os descontos mais altos do ano. Já uma outra versão dada pela Wikipedia, conta que como o feriado de Ação de Graças é numa quinta-feira, alguns estadunidenses costumavam pedir dispensa do trabalho na sexta-feira e aproveitavam um fim de semana de 4 dias, o que enfurecia os patrões reforçando o nome Black Friday.\\
\newline
Longe das origens soturnas que podem ter batizado a data, seguem boas dicas para você aproveitar e economizar nas compras de fim de ano.aa
\\newline
Planeje-se com antecedência\\
\newline
Faça uma lista de tudo o que deseja comprar e tente segui-la;\\
Garanta que seu orçamento comporte a compra de todos os itens da sua lista, não se exceda;\\
Antecipe seu cadastro nos principais sites que vendem os produtos de seu interesse, mas siga as dicas de segurança no fim deste artigo antes de seguir este passo;\\
Cuidado com preços "maqueados", conheça a média de preços daquele produto, durante o ano. Sites como Zoom (https://www.zoom.com.br/) e (http://www.buscape.com.br) mantêm informações históricas de preços dos produtos, vale uma visita antes de aceitar aquela Mega Promoção Imperdível;\\
Antes de fornecer seus dados ou efetivar suas compras pesquise sobre a reputação do site. Utilize serviços como: O do site ReclameAqui (https://www.reclameaqui.com.br/) ou Procon-SP (http://sistemas.procon.sp.gov.br/evitesite/list/evitesites.php). Ainda está na dúvida? Não forneça seus dados nem faça a compra.\\
\newline
Cuidados na hora do pagamento\\
\newline
Prefira, sempre que possível, pagar suas compras com o cartão de crédito, pois em caso de problemas, o estorno é facilitado (devolução do valor pago);\\
Desconfie de sites que aceitem, somente, boletos e transferências bancárias como únicas formas de pagamento;
Confirme se você tem o limite disponível para suas compras naquela data no seu cartão de crédito;
Lembre-se, de considerar o valor a ser pago também incluindo o valor do frete. As vezes o valor baixo do produto foi compensado por valores de frete altíssimos;\\
Ainda com relação ao frete, olho nos prazos de entrega. Nesta época é possível que as previsões ultrapassem 30 dias e você pode ficar sem dar ou receber aquele presente especial de Natal.\\
\newline
Alerta! Dicas de Segurança da Informação e Tecnologia\\
\newline
Dê preferência, a fazer suas compras no conforto de sua casa, onde a rede Wifi é mais segura;
Cuidado com os links recebidos por e-mail, confirme se o que vem após o @ é realmente o domínio da loja que te enviou o anúncio promocional antes de clicar;\\
Se receber um link para imediatamente informar seus dados pessoais, bancários ou do cartão de crédito, desconfie. Você pode ser alvo de roubo de identidade na internet, também conhecido como Phishing;\\
Todo cuidado é pouco. Depois de verificar a reputação do site, ainda fique de olho se você está numa conexão segura. Clique sobre o selo de segurança, que geralmente fica no rodapé da página da web, confirme se o Certificado Digital SSL foi emitido para o mesmo endereço da página em que você está. Além do selo, verifique se existe um cadeado fechado na barra do navegador, clique no cadeado e faça a mesma verificação que fez quando clicou no selo;\\
Após cumprir todas verificações acima e se sentir seguro quanto a reputação do site, faça o seu cadastro;
Guarde todas as informações que se referirem à sua compra. Confirmação de pagamento, código de rastreio, boletos, protocolos de atendimento e exija sempre a nota fiscal estas informações auxiliam na resolução de possíveis problemas.\\
\newline
Outras dicas\\
\newline
Na sexta-feira, muito provavelmente, muitos sites ficarão instáveis e poderão ocorrer  falhas. Não perca a calma, normalmente, o evento se estende por todo o fim de semana até a segunda-feira.
Lembre-se também que: as compras feitas pela internet são amparadas pelo artigo 49 do Código de Defesa do Consumidor que dá o direito ao consumidor de se arrepender e receber os valores eventualmente pagos num prazo de até 07 dias a contar da data da compra.\\
\newline
Siga estas dicas, redobre a atenção nesta época e boas compras!\\
\newline
Disponível em: http://cemignet20/ASI/radar/Lists/Postagens/Post.aspx?List=532c0938-0a20-41c0-be58-5a98329d9953&ID=59&Source=http\%3A%2F%2Fcemignet20%2FASI%2Fradar%2FLists%2FPostagens%2FAllPosts%2Easpx&Web=fd3b1f88-aa18-4ad7-8a9e-582adc0da445
\end{document}