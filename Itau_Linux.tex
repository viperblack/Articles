\documentclass[10pt,a4paper]{article}
\usepackage[utf8]{inputenc}
\usepackage{amsmath}
\usepackage{amsfonts}
\usepackage{amssymb}
\usepackage{graphicx}
\usepackage{hyperref}
\title{Utilizar Internet Banking do Itaú no Google Chrome em qualquer Linux}
\author{Vimerson Pereira da Silva}
\begin{document}
\maketitle
O Itaú utiliza como camadas de segurança um "Aplicativo Guardião" ou um aplicativo que substitue o acesso via browser o problema é que os dois só são compatíveis com o Windows e nem tente utilizar o Wine para esta ignóbil tarefa.\\

Mas descobri, recentemente, que se o seu browser se apresentar para o Itaú, sendo o Google Chrome rodando no sistema operacional BSD, tudo bem... Abrem-se as Portas da Esperança. Veja mais aqui: \href{https://www.vivaolinux.com.br/dica/Acessar-site-do-Itau-Banco-de-qualquer-Linux-2016}{Itaú Bankline via Linux}\\

Mas o procedimento faz o Firefox dizer que é o Chrome. Vamos fazer o Chrome dizer que é ele mesmo, mas rodando no BSD.\\
Para isso precisaremos instalar:\\
User-Agent-Switcher via 
\href{https://chrome.google.com/webstore/detail/user-agent-switcher/lkmofgnohbedopheiphabfhfjgkhfcgf?utm_source=chrome-app-launcher-info-dialog}{Chrome Web Store}.\\

Depois de instalado vá até o ícone da extensão User-agent-switcher e colar a seguinte string \textit{Mozilla/5.0 (X11; NetBSD) AppleWebKit/537.36 (KHTML, like Gecko) Chrome/68.0.3440.75 Safari/537.36} no campo de texto e clicar "Change".</br>
</br>
Atualizado em 26/09/2018.

\end{document}
