\documentclass[12pt,a4paper]{article}
\usepackage[utf8]{inputenc}
\usepackage[portuguese]{babel}
\usepackage[T1]{fontenc}
\usepackage{setspace}
\usepackage{graphicx}
\usepackage[left=3cm,right=2cm,bottom=2cm,headheight=3cm]{geometry}
\usepackage{fancyhdr}
\pagestyle{fancy}
\renewcommand{\headrulewidth}{0pt}
\usepackage[alf]{abntex2cite}
\usepackage[shortlabels]{enumitem}


\lhead{\begin{picture}(0,0){\includegraphics{logo.jpg}} \end{picture}}
\rhead{\footnotesize{\textsf{FACULDADE DE CIÊNCIAS\\EMPRESARIAIS - FACE \\ \textit{ENTREGA DE ATIVIDADE}}}}

\begin{document}
\begin{table}[h!]
	\begin{tabular}{|p{440pt}|}
		\hline
		\textsf{Aluno(a): Vimerson Pereira da Silva}\\
		\hline
		\textsf{Professor(a): Natália Oliveira}\\
		\hline
		\textsf{Tutor(a):}\\
		\hline
		\textsf{Disciplina: Aspectos Tributários Aplicados à Contabilidade}\\
		\hline
		\textsf{Atividade: Atividade 02 - Artigos sobre política e reforma fiscal}\\
		\hline
	\end{tabular}
\end{table}

\onehalfspacing

\noindent
Acesse os artigos relacionados abaixo, e faça uma leitura atenciosa. Em seguida, elabore um texto que sintetize seus conhecimentos.

• Imposto, remendo e reforma escrito por o Estado de São Paulo.

Disponível em: <https://impostometro.com.br/Noticias/Interna?idNoticia=78>.

• O modelo regressivo de tributação no Brasil escrito por Joacir Servegnan.

Disponível em: <http://artigoscheckpoint.thomsonreuters.com.br/a/5s7p/o-modelo-regressivo-de-tributacao-no-brasil-joacir-sevegnani>.\\
\newline

\centerline{Que país é "E\$TE"?}
\noindent
A canção de Renato Russo, conforme a Wikipedia, escrita em 1978, enquanto ele ainda fazia parte da banda: Aborto Elétrico e lançada em 1987 ainda permanece sem resposta definitiva.\\
A letra é uma crítica à corrupção a utopia e a confusão política da época, vivida pelo Brasil.\\
Realmente, acho que ainda não temos resposta para pergunta do poeta. Mas eu me arriscaria a dizer que: "Nós o Brasil". somos um país, fortemente, plutocrático e permitimos que muitos dos princípios que favorecem a concentração de poder e riqueza, segundo Noam Chomsky, demonstrados em seu documentário: "Requiem for the American Dream" de 2016.  aconteçam e se estabeleçam desde o nosso descobrimento em 1500:
\begin{enumerate}
\item Reduzir a Democracia
\item Moldar a ideologia
\item Redesenhar a economia
\item Deslocar o fardo de sustentar a sociedade para os pobres e classe média
\item Atacar a solidariedade
\item Deixar reguladores atuarem em causa própria
\item Controlar as eleições
\item Manter o povo na linha
\item Criar e propagar o consumismo
\item Marginalizar a população
\end{enumerate}
A política regressiva de imposto, a tributação da produção e não da concentração de riqueza, a apatia na regulação e na fiscalização, a polarização política de forma odienta, o alto grau de endividamento da Sociedade e até dos Estados, mostram que o documentário, aplicado ao cenário do Brasil, tem mais de realidade do que ficção e mais fatos do que teorias conspiratórias.\\
Mesmo a política fiscal brasileira não cumprindo, a contento, nenhuma das suas principais funções:
\begin{itemize}
\item Alocativa, que diz respeito ao fornecimento de bens públicos; 
\item Estabilizadora que tem por objetivo o uso da política econômica visando a um alto nível de emprego e, por fim;
\item Distributiva, que visa a redução da desigualdade social através da redistribuição de renda.
\end{itemize}
Vivemos novamente uma utopia otimista de que tudo dará certo, ou nas palavras de Renato:\\ "Mas o Brasil vai ficar rico\\
 Vamos faturar um milhão..."\\
A necessidade de reforma é um imbróglio de governança corporativa, onde há claramente um conflito de agências, plutocracia versus democracia, sendo a plutocracia mantida pela marginalização do povo, minando sua solidariedade, insinuando e propagandeando de que manter a "Ordem e o Progresso" é melhor do que ser o "Brasil, um país de todos".\\
A modernização tributária e fiscal, profunda, é indesejada pelos que faturam e que em rede nacional nos prometem que ficaremos ricos durante a sua gestão política. Vamos vivendo de remendos que mantêm o "status quo" e com prazo de validade político.
\end{document}
